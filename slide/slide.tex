\documentclass{beamer}

\usepackage{beamerutc}

\usetheme{utc}
%\usetheme[shadowed=true]{utc}
%\usetheme[rounded=false]{utc}
% Option utcrecherche pour les thèses, les présentations d'articles…
% \usetheme[utcrecherche=true]{utc}
% Option pinkie pour les relations internationales
% \usetheme[pinkie=true]{utc}
%\usetheme[navigation=true,rounded=false]{utc}
%\usetheme[navigation=vertical,rounded=false]{utc}
%\usetheme[withsecplan=true,navigation=vertical,rounded=false]{utc}
% You cannot have rounded=false and shadowed=true in the same time.
% \usetheme[navigation=false, option=withsecplan]{utc}

% default is equivalent to the following
%\usetheme[withsecplan=false,navigation=false,rounded=true,shadowed=false,utcrecherche=false]{utc}

\title{Essai de presentation}
%\subtitle{Test de sous titre}
\author{Étienne Deparis}
\email{etienne@depar.is}
%\institute{Présentation R\&D}
\date{16 juillet 2010}
\uv{XX42}
\semestre{P10}

\begin{document}

\frame[plain]{\utctitlepage}

\section{Qu'est-ce que la vie ?}

\begin{frame}{Pourquoi le ciel ?}
  \begin{itemize}
  \item \textsc{Beamer}, c'est vachement tout beau!
  \item avec de nombreux atouts partout
  \item et même là où on ne les attends pas !
  \end{itemize}
\end{frame}

\begin{frame}
  Et c'est facile.
\end{frame}

\section{Le contenu de ma prez}
\subsection{C'est du gâteau}
\begin{frame}{C'est du gâteau}
  \begin{itemize}
	\hilite<1>\item Cro bien
	\hilite<2>\item yep yep !
  \end{itemize}
\end{frame}

\subsection{Les blocs prédéfinis}
\begin{frame}{Les blocs prédéfinis}
  % Un bloc normal pour comparer
  \begin{block}{Un bloc normal}
    Texte du block normal
  \end{block}

  % Un bloc "example"
  \begin{exampleblock}{Un bloc example}
    Exemple de block example
  \end{exampleblock}

  % Un bloc "alert"
  \begin{alertblock}{Un bloc alert}
    Texte du block alerte
  \end{alertblock}

\end{frame}

\begin{frame}{Les blocs prédéfinis}
  \begin{definition}
    environnement definition
  \end{definition}

  \begin{example}
    environnement example
  \end{example}

  \begin{proof}
    environnement proof
  \end{proof}

  \begin{theorem}
    environnement theorem
  \end{theorem}
\end{frame}

\section{Conclusion}

\begin{frame}{Conclusion}
  \begin{block}{}
    C'est top!
  \end{block}
\end{frame}

\frame[plain]{\utcthankspage[Merci !]}
% \frame[plain]{\utcthankspage} % Default print Thanks

\end{document}

%%% Local Variables:
%%% mode: latex
%%% TeX-master: t
%%% End:
